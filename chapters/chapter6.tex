\section{Kaon Beam Spin Asymmetry}

In addition to the unpolarized structure functions which will be accessible from the SIDIS cross section measurement, the beam spin asymmetry for positively charged k-mesons in SIDIS is measured.  This gives us access to an additional structure function $F_{LU}^{\sin\phi}$.  The measurement was performed as a function of $\phi_h$ and 4 different kinematic variables $x$, $Q^2$, $z$, and $P_T$.  \\

In order to measure kaons in the kinematically accepted DIS region, we exclude events with have $Q^2 < 1 \k GeV^2/c^2$ or which have $W < 2 \; GeV/c^2$.  Additionally when measuring the axes ($x$, $Q^2$, and $P_T$), the range of $z$ values included in the sample is limited to be within 0.25-0.75.  This cut is used as an attempt to make our measurement in the current fragmentation region where factorization has been proved.  

\subsection{Helicity Determination}
The beam helicity is measured periodically with a Moller polarimeter.  These measurements were analyzed statistically by Wes Gohn and it was observed that the average beam polarization was 75\%, with a variance of approximately 1\%.  

\subsection{Measurement of $\phi$ Dependent Asymmetry}
After the final sample of events has been selected, the events are binned in 12 bins of $\phi$ and 10 bins of the kinematic axis.  The calculation of the BSA in each bin is done.

\begin{equation}
  A = \frac{1}{P_e} \frac{\Delta N}{N}
\end{equation}

Here $P_e$ is the fractional polarization of the beam (explained above), $N = N_+ + N_-$ and, $\Delta N = N_+ - N_-$.  By using error propagation and assuming the statistical error on the counts $N$ to be $\sqrt{N}$ one can show that the uncertainty in the beam spin asymmetry due to the statistical uncertainty in the counts is,

\begin{equation}
  \sigma_A^2 \approx \frac{A^2}{P_e^2} \sigma_{P_e}^2 + \frac{4}{P_e^2 (N_+ + N_-)^4}(N_-^2 \sigma_{N_-}^2 + N_+^2 \sigma_{N_+}^2) 
\end{equation}

where the uncertainty due to the $P_e$ will be added to the systematic errors.  The counts are taken to be drawn from a Poisson distribution (which describes the probability to observe $\nu$ events when you expect N).  The variance on the counts is equal to the mean (N).  The statistical errors are then quoted as shown below. 

\begin{equation}
  \sigma_A^2 \approx \frac{4N_+N_-}{P_e^2 (N_+ + N_-)^3} 
\end{equation}

\subsection{Systematic Uncertainties}

The value of the measurement of the beam spin asymmetry in each bin can be influenced by many factors.  Common examples include the particle identification routines, the subtraction of backgrounds, and the calibration (mis-calibration) of detectors.  A mis-calibrated detector represents a systematic effect, that consistently moves the measured value one direction or another.  These effects are corrected for, and the detector calibration is adjusted or fixed.  In this way I  have attempted to eliminate all systematic effects.  The degree of uncertainty which remains as to my ability to correct for a systematic effect, I call systematic uncertainty. \\

First the contribution to the total systematic uncertainty which comes from the uncertainty in the measured value of the beam polarization is considered.  This contribution is combined together with other sources of systematic uncertainty, such as the uncertainty on input parameters to the analysis code.  The complete procedure for handling systematic uncertainties will be described in the thesis work, and my final estimates will be tabulated.  My current estimates are shown as red bars in the figure below.

\easyFigure{image/bsa_z_sys.pdf}{The beam spin asymmetry as a function of $\phi_h$ between -180 and 180 degrees shown for 10 bins of $z_h$, increasing from top left to bottom right.}

\subsection{Conclusion and Outlook}
We have extracted integrated beam spin asymmetry measurements for four different kinematic variables.  In doing so, we observe non-zero contributions from twist-3 (or higher) TMD/FF functions.  Our measurement can now be studied using phenomenological models of TMD and FF distributions.  To conclude this project we will provide systematic uncertainties for the extracted modulations, and compare it to a phenomenological model.

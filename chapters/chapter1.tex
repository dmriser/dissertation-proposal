\begin{abstract}

This dissertation will present measurements of the differential cross section for $e p \rightarrow e \pi^{0} X$ over a large kinematic range ($0.1 \leq x \leq 0.6$, $1.0 \leq Q^{2} \leq 4.8$, $0 \leq P_{T}^{2} \leq 1$).  Our analysis is performed on data taken by the CEBAF Large Acceptance Spectrometer (CLAS) at Jefferson Lab (JLab) during the E1-F rungroup in 2003.  This experimental run was performed at beam energy $E_{Beam} = 5.498$ GeV with longitudinally polarized electrons which were incident on unpolarized liquid hydrogen (nearly stationary protons).  This cross section measurement will supply information about the transverse momentum dependent parton distribution functions (TMD's).

\end{abstract}

\section{Motivation}
Quantum Chromodynamics (QCD) is generally accepted as the correct theory to describe the strong interaction between colored particles (quarks and gluons).  The classical QCD Lagrangian is obtained by simply asking that the Dirac action remain locally phase invariant in an $SU(3)$ color space (these ideas were developed by Gell-Mann and Zwieg) \cite{zwieg}.  However, the non-Abelian character of the generators of $SU(3)$ complicate QCD, introducing interactions between the gluons which mediate the strong force (direct photon-photon coupling is not preset in Quantum Electrodynamics).  Quantum Chromodynamics is a rich theory, which displays varied  behaviour over a wide range of energies, such as confinement and asymptotic freedom.  The belief in QCD comes from a wealth of experimental data from lepton-lepton, lepton-hardon, and hadron-hadron colliders, where perturbative calculations can be tested at high energies.  It is really the quest to understand proton structure which motivates this dissertation, which by way of comparison with phenomenological models may also give insight into the confinement mechanism in QCD \cite{tung}. 






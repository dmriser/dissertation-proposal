\section{Conclusion}
%In this document, I have described my ongoing research efforts to make an impact in the field of nucleon structure research.  Specifically, we propose that this dissertation project will (1) extract the %structure function $F_{LU}^{\sin\phi}$ for positive kaons, (2) take steps necessary to extend a current CLAS measurement of unpolarized SIDIS azimuthal modulations to the absolute (SIDIS) cross %section, and (3) utilize existing phenomenology and modeling to predict these measurements.  My work on both (1) and (3) is now at an advanced stage, and the remaining tasks have been briefly %described in the document above.  

This proposal outlines the two primary goals of my thesis research, the measurement of unpolarized SIDIS cross sections for $\pi^{\pm}$ and the measurement of beam spin asymmetries for the positively charged k-meson in SIDIS.  Recently, much emphasis has been placed on the importance of understanding the transverse spatial and momentum structure of the quarks that comprise the nucleon, our work contributes significantly in understanding TMD distributions in protons.  Our measurement of cross sections in SIDIS provides the opportunity for directly accessing the structure functions $F_{UU}^{\cos\phi_h}$ and $F_{UU}^{\cos(2\phi_h)}$ which have previously only been accessed through ratios.  This mature work is the result of years of collaborative effort, and is now in the final stages of preparation.  My measurement of a non-vanishing beam spin asymmetry in SIDIS provides an important reminder of the importance of higher twist contributions in TMD physics at Jefferson lab energies, at least to the structure function $F_{LU}^{\sin\phi_h}$.  A complete report summarizing the beam spin asymmetry results is now being reviewed for submission as a CLAS analysis note.  Mapping of quark dynamics within protons and neutrons is a large and collaborative task, and I feel that this work represents an important building block.  It is my hope that the approval committee agrees with this statement, and approves this dissertation for completion. 

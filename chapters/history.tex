
\section{History}

For thousands of years humans have been interested in finding the smallest building blocks of matter.  In the $5^{th}$ century B.C. Greek philosophers Leucippus and Democritus proposed all the matter in existence was composed of tiny indivisible chunks called atoms.  We have since learned (amazingly) that matter consists of just about 90 different atoms, which were categorized by Mendelev.  Even more amazing was the fact that these 90 or so atoms could be explained with just 3 building blocks, electrons, protons, and neutrons (the role of each of these in the atom is now clear).  The early 1920's saw the development of quantum mechanics, and Born's statistical interpretation of the wavefunction $\Psi(\vec{x},t)$ of Schrodinger.  It was then possible to gain understanding of the structure of atoms. \\

In the 1950's and 1960's the experimental work of Hoftstader demonstrated that the proton itself was not pointlike.  At the same time at the Stanford Linear Accelerator (SLAC), a ``zoo'' of new particles which did not seem to make up the atom nor fit into the table of Mendelev were being discovered at an alarming rate.  Gell-Mann and Zweig proposed in 1964 that the proton, neutron, and most of the list of particles discovered belonged to a family of particles called Hadrons.  They went on to explain that these Hadrons should be composed of smaller fractionally charged particles called quarks.  Just as matter had been reduced to atoms and atoms had been reduced to electrons, protons, and neutrons, the hadrons had now been reduced to a set of more fundemental particles.  However, it was clear that fractionally charged particles had never been seen, and creativity would be needed in order to study these quarks.  \\

The quark-parton model, inspired by Feynman was formalized by Bjorken in the late 1960's.  The quark-parton model provides a means for measuring the structure of the proton (hadrons) in terms of the internal (parallel to the collision direction)  momentum of the quark struck in the collision.  The parton distribution functions of the quark-parton model $f_{a}(x,Q^{2})$ at leading order depend only on the momentum fraction of the struck quark $x$ and can be interpreted as the probability for finding a quark with flavor $a$ in the hadron with momentum fraction $x$.  The quark-parton model was largely succesful, but fails to provide a complete picture of the structure of hadrons.  \\

In recent years, Drell-Yan ($hh \rightarrow l \bar{l}X$) and SIDIS have been used to measure Transverse Momentum Dependent Parton Distribution Functions (TMD's), which depend on the 3-momenta of the quark in the hadron.  Exclusive processes such as deeply virtual Compton scattering (DVCS) and deeply virtual meson production (DVMP) have been used to access Generalized Parton Distribution Functions (GPD's) which depend on the longitudinal momentum fraction, and the position of the quark in the transverse plane, usually denoted $\vec{b_{\perp}}$.  These two different sets of functions can be combined to give the Generalized Transverse Momentum Distributions (GTMD's). \\






